\documentclass[a4paper, 11pt]{article}
\usepackage{amsmath}
\usepackage[ngerman]{babel}
\usepackage{paralist}
\usepackage[utf8]{inputenc}
\usepackage{fullpage}


\title{Zusammenfassung Einführung in die Analysis}
\author{Jannis Hübl und Tobias Bechtold}
\date{Juli 2013}


\begin{document}

\maketitle
\newpage
\tableofcontents
\newpage
\section{Konvergenzkriterien für Reihen (S. 11)}

\subsection{Quotientenkriterium}


$ c = \lim_{x \to \infty} |\frac {a_{k+1}} {a_{k}}|$


\subsection{Wurzelkriterium}

$ c = \lim_{x \to \infty} \sqrt[k]{|a_{k+1}|} = 0 $

\subsection{Divergenz/Konvergenz}

$ c < 1 => $Absol. Konvergenz

$ c > 1 => $ Divergenz

$ c = 0 => $Keine Aussage möglich

\section{Konvergenzkriterium für Potenzreihen (S. 12)}

Bsp.: $ \sum \frac{2k}{3^k} (x + 2)^k$

\subsection {Substituieren}
u = x + 2

\subsection {r bestimmen}

$ r = \lim_{k \to \infty} |\frac{a_k}{a_{k+1}}| $

oder

$ r = \frac {1} {\lim_{k \to \infty} \sqrt[k]{|a_k|}}$

$Bsp: r = \lim_{k \to
\infty} \frac {\frac {2k} {3^k}}{\frac{2(k+1)}{3^{k+1}}} = \lim_{k \to \infty}
\frac {2k 3^k} {2(k+1) 3^{k+1}} = 3$

\subsection { u bestimmen}

$u_0 = -r = -3$

$u_1 =  r = 3$

\subsection { Ränder auf Divergenz/Konvergenz prüfen}

$u_0 = -3: \sum \frac{2k} {3^k} (-3)^k => Div.$

$u_1 = 3: \sum \frac{2k} {3^k} (3)^k => Div.$    

$=> u \in (-3, 3)$

\subsection { Konvergenzgebiet ausrechen}

$u = (x - 2)$

$u_0 = -3 = (x_0 - 2) => x_0 = -5$

$u_1 = -3 = (x_1 - 2) => x_1 = 1$

$=> x \in (-5, 1)$ 


\section{Ableitungen}

TODO: Tabelle mit Ableitungen, Kettenregel, Quotientenregel mit ein paar
schweren beispielen, Richtungsableitungen, usw\ldots 
\subsection{Ableitungsregeln}
\renewcommand{\arraystretch}{1.5}

\begin{tabular}{l | r} 
	$f(x)$ & $f'(x)$\\
	\hline 
	$c = const$ & $0$\\
	$x^n$ & $n \cdot x^{n-1}$\\
	$\sqrt{x}$ & $\frac{1}{2\cdot\sqrt{x}}$\\
	$e^x$ & $e^x$\\
	$a^x$ & $\ln{a} \cdot a^x$\\
	$\frac{1}{x^n}$ & $- \frac{n}{x^{n+1}}$\\
	$\sqrt[n]{x}$ & $\frac{1}{n \cdot \sqrt[n]{x^{n-1}}}$\\
	$\ln{x}$ & $\frac{1}{x}$\\
	$\sin{x}$ & $\cos{x}$\\
	$\cos{x}$ & $-\sin{x}$\\
	$\tan{x}$ & $\frac{1}{\cos^2{x}}$\\
\end{tabular} \newpage

\subsection{Produktregel}
Die Ableitung von $f \cdot g$ ist gegeben durch: \newline
\newline $(f(x) \cdot g(x))' = f'(x) \cdot g(x) + f(x) \cdot g'(x)$\newline
\newline Beispiel: $f(x) = (5x^2 - 3) \cdot (8x^3+2x)$

\begin{tabular}{l r}
	$f(x) = 5x^2 - 3$ & $f'(x) = 10x$\\
	$g(x)=8x^3+2x$ & $24x^2 + 2$\\
\end{tabular} \newline
\newline Produktregel anwenden: 
\newline $f'(x)=10x \cdot (8x^3+2x) + (5x^2-3) \cdot (24x^2+2)$ \newline
\newline Vereinfachen: $f'(x) = 200x^4-42x^2-6$

\subsection{Quotientenregel}

Die Ableitung von $\frac{f}{g}$ ist gegeben durch: \\
\newline $(\frac{f(x)}{g(x)})' = \frac{f'(x) \cdot g(x) + f(x) \cdot
g'(x)}{(g(x))^2}$ \\

Beispiel: $(\frac{5x}{e^x})' = \frac{5e^x-5xe^x}{e^{2x}} = \frac{5(1-x)}{e^x}$

\subsection{Kettenregel}

$(f(g(x)))' = f'(g(x)) \cdot g'(x)$\\
\newline Beispiel 1: $e^{x^{2}}$\\ 
\begin{tabular}{l r}
	$f(x)=e^x$ & $f'(x)=e^x$\\
	$g(x)=x^2$ & $g'(x)=2x$\\
	$f'(g(x))=e^{x^{2}}$ & $g'(x)=2x$\\
	& \\
	$f(g(x))'=2xe^{x^{2}}$
\end{tabular} \newline
\newline Weitere Beispiele: \\
$(e^{2x})' = 2e^{2x}$\\
$(\sin{e^{2x}})' = \cos{e^{2x}} \cdot 2e^{2x}$\\
$(2^{\cos{x}})'=\ln{2} \cdot 2^{\cos{x}} \cdot (-\sin{x})$

\subsection{Partielle Ableitung}
Es wird immer nur nach einem Parameter abgeleitet.\\
Beispiel: $f(x,y)= x^2 + y^2$ \\
\newline \begin{tabular}{l r}
	$\frac{\partial f(x,y)}{\partial x}= 2x$ & $\frac{\partial f(x,y)}{\partial 
	y}=2y$\\
\end{tabular}

$\frac{\partial f}{\partial x} = 
\begin{pmatrix}
	\frac{\partial f}{\partial x} \\
	\frac{\partial f}{\partial y}
\end{pmatrix}
= 
\begin{pmatrix}
	2x\\
	2y
\end{pmatrix}$ \newline

\section{Fehlerrechnung (S. 34)}

Bsp: $u(t) = u_0 e^{- \frac{t}{R C}}; u_0 = 5V; R = 100 \Omega; C =
10^{-3}F; max. Fehler: 0,01 $
\subsection { max. Fehler der ``Eingangswerte'' berechnen }

$ \Delta C = C * maxFehler = C * 0,01 = 10^{-3} * 0,01 = 10^{-7} $

$ \Delta R = R * maxFehler = R * 0,01 = 100 * 0,01 = 1 $

\subsection { Ableiten nach Fehlervariablen}

$\frac {du} {dR} = u_0 e^{\frac{-t}{RC}} (\frac{t}{C}R^{-2}) = \frac{u_0 *
t}{R^2 * C} e^{\frac{-t}{RC}}$

$\frac {du} {dC} = \frac{u_0 *
t}{R * C^2} e^{\frac{-t}{RC}}$

\subsection { Absoluterfehler ausrechnen}

Werte in Formel einsetzen und ausrechnen

$\Delta u \leq |\frac {du} {dR} \Delta R |+| \frac {du} {dC} \Delta C| = |
0,0184 * 1 |+| 183939,72 * 10^{-7}|$

\subsection { Relativerfehler}

Absoluterfehler / Wert an einer Stelle

(die stelle soll 1ms sein)

$\frac{\Delta u}{u(1ms)} = \frac {0,0184} {1,84} + \frac {18393,72* 10^{-7}}
{1,84} = 0.02 $

\newpage
\section{Extremalaufgaben (Methode von Lagrange) (S. 38)}
(Lagrange siehe S. 40 + 41)

$Bsp: Q = y - x^2; Restriktion: y = 2x - 1 $

\subsection{Graph mit Höhenlinien zeichnen}

Für Q Werte einsetzen (wenn nicht vorgegeben -1, 0, 1.. irgendwelche beliebigen
``schönen'' Werte) und dann immer nach y auflösen, dann die erhaltene Funktionen
zeichnen. (Dazu die Q Werte schreiben, das sind die Höhenwerte). Im folgenden
Beispiel lautet die Funktion $Q = y - x^2$. Man bekommt nun sozusagen Parabeln
raus, die man dann einfach in ein Koordinatensystem einzeichnet. \newline

$Q = -1 => -1 = y - x^2 <=> y = x^2 -1$

$Q = 0 => 0 = y - x^2 <=> y = x^2$

$Q = 1 => 1 = y - x^2 <=> y = x^2 +1$

$Q = 2 => 2 = y - x^2 <=> y = x^2 +2$


\subsection{Gradienten in Punkten bestimmen}

Dazu die Funktion Partiell ableiten und in die Ableitung den Punkt einsetzen,
das Ergebnis der Gradient in dem Punkt \newline
Bsp: $\frac{\partial Q}{\partial x} = 
\begin{pmatrix}
	\frac{\partial Q}{\partial x} \\
	\frac{\partial Q}{\partial y}
\end{pmatrix}
=  
\begin{pmatrix} 
	-2x-1 \\
	1
\end{pmatrix}$ \newline
\newline
Punkt (0/1) einsetzen =$>$ 
$\begin{pmatrix}
	0\\
	1
\end{pmatrix}$ 

\subsection{Optimierungsproblem lösen}

An welchem Punk ist die Restrikion am Höchsten/Niedristen? \newline

$L(x,y,\lambda) = Q(x,y) + \lambda * h(x,y) = y-x^2+\lambda(y-2x+1)$ \newline

\begin{compactenum}[(1)] 
\item $\frac{\partial L}{\partial x} = 0 = -2x-2\lambda$ 
\item $\frac{\partial L}{\partial y} = 0 = 1 + \lambda => \lambda = -1 => x = 1
	=> y = 1$
\item $\frac{\partial L}{\partial \lambda} = 0 = y-2x+1$ \newline
\end{compactenum}

im Punkt (1/1) ist ein Maximum \newline
\newline (Das findet man heraus wenn man den Punkt in die Zeichnung einsetzt und betrachtet wie die Restiktion auf der Fläche verläuft.)

\section { Taylor-Approximation (S. 52 -54)}

Approximation $\approx $Annäherung an eine Andere Funktion \newline 
\newline Bsp: $f(x) = x^3 + x^2 + 3x - 7 ; x_0 = 1$ 	

\subsection {Welchen Grad soll die Zielfunktion haben?}

$x^3; x^2; x^n ?$\newline
\newline In unserem Bsp. soll die Zielfunktion vom Grad 2 sein.


\subsection {Start Funktion n-mal ableiten.}

f(x) 2 mal ableiten\newline
\newline
$f'(x) = 3x^2 + 2x + 3$\newline
$f''(x) = 6x + 2$


\subsection {Einsetzen in Taylorsche Formel}

$ f(x) = \sum_{k=0}^n \frac {f^{(k)}(x_0)}{k!}(x - x_0)^k$\newline
\newline Bsp:\newline				
$f(x) = \frac {f(x_0)}{0!} (x - x_0)^0 + \frac {f'(x_0)}{1!} (x - x_0)^1 + \frac {f''(x_0)}{2!} (x - x_0)^2 =$\newline
\newline $\frac {x_0^3 + x_0^2 + 3x_0 - 7}{0!} (x - x_0)^0 + \frac {3x_0^2 + 2x_0 + 3}{1!} (x - x_0)^1 + \frac {6x_0 + 2}{2!} (x - x_0)^2 = $\newline
\newline $\frac {1^3 + 1^2 + 3 * 1 - 7}{0!} (x - 1)^0 + \frac {3 *1^2 + 2*1 + 3}{1!} (x - 1)^1 + \frac {6 * 1 + 2}{2!} (x - 1)^2 = $\newline
\newline $-2+8(x-1)+\frac 8 2 (x-1)^2 = 4x^2 - 6$

\subsection {Nullstellen berechnen}

In der Probeklausur sollten wir die Nullstellen dieser Funktion berechen. \newline
fals jemand damit ein Problem hat, hier die Lösung: \newline
\newline $ 4x^2 - 6 = 0 => x= \pm \sqrt {\frac 3 2}$  

\section {Differenzialgleichungen S. 71}

( Man will eine Formel Ableiten in der eine eigenen Ableitung vorkommt )\newline
\newline Bsp.: $ y''' - 4y'' +5y' -2y = 0 ; y(0) = -1; y'(0) = -2; y''(0) = 5;$

\subsection { Charakteristische Gleichung aufstellen} 
Alle $y^{(k)}\; durch \; \lambda^k ersetzen.$\newline
\newline Bsp: $ \lambda^3 - 4 \lambda^2 + 5 \lambda^1 - 2\lambda^0 = \lambda^3 - 4 \lambda^2 + 5 \lambda - 2$
\subsection {Nullstellen der Charakteristeschen Gleicheung berechnen}

(Die sollten normalerweise vom Makus angegeben sein) \newline
\newline

\subsection {Basislösung zuordnen}

Die Basislösungen sind Hier:\newline
\begin{tabular} {l | c | r}
	Bsp & Lösung der char. Gleichung &  Basislösung der Dgl. \\
	\hline
	$(x-2) $ & $\lambda$ 1-fach reell & $e^{\lambda x} $ \\
	$(x-2)^2$ & $\lambda$ k-fach reell & $ e^{\lambda x }, x e^{\lambda x},..., x^{k-1} e^{\lambda x}$ \\  
\end{tabular}

\end{document}
